% Adapted from Alex Reustle's CMSC351 Course Notes

% This program is free software: you can redistribute it and/or modify
% it under the terms of the GNU General Public License as published by
% the Free Software Foundation, either version 3 of the License, or
% (at your option) any later version.

% This program is distributed in the hope that it will be useful,
% but WITHOUT ANY WARRANTY; without even the implied warranty of
% MERCHANTABILITY or FITNESS FOR A PARTICULAR PURPOSE.  See the
% GNU General Public License for more details.

% You should have received a copy of the GNU General Public License
% along with this program.  If not, see <http://www.gnu.org/licenses/>.
\documentclass[english, 10pt]{article}

\usepackage{notes}
\usepackage{inconsolata}
\usepackage[shellescape]{gmp}
\allowdisplaybreaks%
\newcommand{\thiscoursecode}{CMSC 216}
\newcommand{\thiscoursename}{Introduction to Computer Systems}
\newcommand{\thisprof}{Dr.\ Ilchul Yoon}
\newcommand{\me}{Akilesh Praveen}
\newcommand{\thisterm}{Spring 2020}
\newcommand{\website}{http://cs.umd.edu/class/spring2020/cmsc216/}%chktex 8
\usepackage{ifpdf}
\ifpdf%
\DeclareGraphicsRule{*}{mps}{*}{}
\fi
% \listfiles

\usepackage[utf8]{inputenc}
 
\usepackage{listings}
\usepackage{xcolor}
 
\definecolor{codegreen}{rgb}{0,0.6,0}
\definecolor{codegray}{rgb}{0.5,0.5,0.5}
\definecolor{codepurple}{rgb}{0.58,0,0.82}
\definecolor{backcolour}{rgb}{0.95,0.95,0.94}
\definecolor{codered}{rgb}{0.5,0.15,0.15}
\definecolor{commentred}{rgb}{1,0.01,0.02}
 
\lstdefinestyle{mystyle}{
    backgroundcolor=\color{backcolour},   
    commentstyle=\color{codegreen},
    keywordstyle=\color{red},
    numberstyle=\tiny\color{codegray},
    stringstyle=\color{codered},
    basicstyle=\ttfamily\footnotesize,
    breakatwhitespace=false,         
    breaklines=true,                 
    captionpos=b,                    
    keepspaces=true,
    xleftmargin=.15\textwidth,
    xrightmargin=.15\textwidth,
    linewidth=\textwidth,                 
    numbers=left,                    
    numbersep=5pt,                  
    showspaces=false,                
    showstringspaces=false,
    showtabs=false,                  
    tabsize=2,
    belowskip=3em,
    aboveskip=3em,
}

\lstset{style=mystyle}


% \VerbEnvir{align tikzpicture algorithm}
%%%Headers
\chead{216-Introduction to Computer Systems}
\lhead{\thisterm}

%%%%% TITLE %%%%%
\graphicspath{{../}}
\newcommand{\notefront}{%
\pagenumbering{arabic}
\begin{center}
{\small}
\textbf{\Huge{\noun{\thiscoursecode}}}
{\Huge \par}
{\Large{\noun{\thiscoursename}}}\\
\vspace{0.1in}
\vspace{0in}\includegraphics[scale=0.3]{umd_cs.jpg} \\
\vspace{0.1in}{\noun\me} \\
{\noun\thisprof} \ $\bullet$ \ {\noun\thisterm} \ $\bullet$ \ {\noun{University of Maryland}} \\
{\ttfamily \url{\website}} \\
\end{center}
}

 \tikzstyle{class}=[
    rectangle,
    draw=black,
    text centered,
    anchor=north,
    text=black,
    text width=2cm,
    shading=axis,
    bottom color={rgb:red,222;green,222;blue,222},
    top color=white,shading angle=45]

\begin{document}
% \renewcommand\familydefault{\sfdefault}
% \sffamily
  % Notes front
  \notefront%
  % Table of Contents and List of Figures
  \tocandfigures%
  
\section{Notes}

This is a compilation of my notes for CMSC216 as a TA for the Spring 2020 offering of the course at the University of Maryland. All content covered in these notes was created by Dr. Ilchul Yoon and Dr. A.U. Shankar at the University of Maryland, and these are simply my transcriptions of the content provided in lecture and lab, along with my additional insight from the course compiled in a \LaTeX document.
\newline

The notes template in use is Alex Reustle's template, which can be found on his github at the following location: \texttt{https://github.com/Areustle/CMSC351SP2016FLN}
\linebreak

Please send errors to \texttt{apraveen@cs.umd.edu}
  
  
\section{Introduction to CMSC216}

CMSC216 is where you learn how a computer works on a much lower level than you've experienced before. There are 3 main components that the course will explore.

\subsection{Overview}

\begin{itemize}
	\item \textbf{UNIX} Threads, processes, and pipes as the building blocks of much bigger applications. We will be working with the UNIX operating system on the development environment at \texttt{grace.umd.edu}
	\item \textbf{C} is a high-performance language that works at a much lower level than Java. Things like memory management and advanced data structures are left up to the user. We'll cover concepts like memory management, pointers, and system calls.
	\item \textbf{Assembly} is even lower-level than C, and studying it will reveal how processors process instructions, store data, and maintain a stack and a heap. It's the lowest level you'll go in this class. For this semester's 216, you will be using MIPS assembly.
\end{itemize}

\subsection{Debugging}

Debugging is accomplished using two main tools: \textbf{Valgrind} for memory related issues, and \textbf{gdb} for general debugging. Both of these tools are very powerful if used correctly, and are explained as the semester progresses.

For more information, check out this link provided in the course slides:
\newline
\newline
\texttt{\href{http://www.cs.usfca.edu/~parrt/doc/debugging.html}{The Essentials of Debugging}}



  
\section{A Main Section (Template)}

Here are some cool notes that will no doubt be very helpful to students.
\newline\newline
Here is some cool code in Maryland colors!

{\centering
\begin{lstlisting}[language=C]
int main() {
	int x = 1;
	int y = 2;
	printf("hello world");
	return x + y;
}
\end{lstlisting}
}


\subsection{A Subsection}
A subheading's text
  
\end{document}